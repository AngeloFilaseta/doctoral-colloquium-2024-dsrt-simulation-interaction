\documentclass[conference]{IEEEtran}
\IEEEoverridecommandlockouts
% The preceding line is only needed to identify funding in the first footnote. If that is unneeded, please comment it out.
\usepackage{cite}
\usepackage{amsmath,amssymb,amsfonts}
\usepackage{algorithmic}
\usepackage{graphicx}
\usepackage{textcomp}
\usepackage{xcolor}
\usepackage{acronym}
\usepackage{hyperref}
\usepackage{cleveref}

\acrodef{API}{Application Programming Interface}
\acrodef{AR}{Augmented Reality}
\acrodef{ACL}{Anti-Corruption Layer}
\acrodef{CLI}{Command Line Interface}
\acrodef{HCI}{Human-Computer Interaction}
\acrodef{PC}{Personal Computer}
\acrodef{HMI}{Human-Machine Interface}
\acrodef{IDE}{Integrated Development Environment}
\acrodef{LLM}{Large Language Model}
\acrodef{GUI}{Graphical User Interface}
\acrodef{UI}{User Interface}
\acrodef{YAML}{YAML Ain't Markup Language}
\acrodef{UX}{User Experience}

\def\BibTeX{{\rm B\kern-.05em{\sc i\kern-.025em b}\kern-.08em
    T\kern-.1667em\lower.7ex\hbox{E}\kern-.125emX}}
\begin{document}

\title{Evolving the Interaction with Simulators\\
%\thanks{Identify applicable funding agency here. If none, delete this.}
}

\author{\IEEEauthorblockN{Angelo Filaseta}
\IEEEauthorblockA{\textit{Department of Computer Science and Engineering} \\
\textit{Alma Mater Studiorum---Università di Bologna}\\
Cesena, Italy \\
0009-0004-6797-6814}
}

\maketitle

\begin{abstract}
    Designing effective and user-friendly Human-Machine Interfaces for simulators can be a complex endeavor that requires adherence to guidelines
    established through decades of research.
    The inherent complexity of the systems under examination contributes to the difficulty of creating intuitive interfaces that suit the user's needs.
    %
    Human-Computer Interactions can be particularly complex for general-purpose simulators,
    since the elements to observe and interact with can drastically change depending on the context.
    %
    Some additional challenges arise when dealing with distributed simulations,
    as the observer might want to interact with multiple simulations at once.
    %
    In this paper,
    we will explore the challenges of designing innovative ways to interact with distributed general-purpose simulations,
    trying to address the issues caused by the inevitable complexity portrayed by these systems.
    %
    Moreover,
    we will also discuss the potential benefits and drawbacks of using Large Language Models to assist the user in interacting with simulations.
\end{abstract}

%\begin{IEEEkeywords}
%    human computer interaction, distributed simulation
%\end{IEEEkeywords}

\subsection{Introduction}
\ac{HCI} focuses on the design,
evaluation,
and realization of \emph{interactive} computing systems for human use.
The effectiveness of a \ac{HMI} heavenly depends on factors that consider both \emph{usability} and \emph{functionality}~\cite{Sinha2010}.
%
The two aspects must be balanced to create easy-to-use interfaces
while providing all the necessary tools for users to reach their goals.
%
The development of interactive interface design has rapidly progressed over the past decades,
significantly focusing on the enhancement of \acp{GUI}~\cite{Murad2019}.
%
However,
research is being conducted to explore new ways to interact with machines,
with inclusions regarding eye-tracking,
hand gesture recognition,
and \acp{LLM}~\cite{Poole2006, Sarma2021, kapania2024imcategorizingllmproductivity}.

As a matter of fact,
observability is a key aspect of simulations,
and a graphical representation is really helpful to visualize the behavior of the system.
%
Consequently,
simulators usually provide \acp{GUI} or some sort of rendering system that allow users to visualize the simulation progress.
%
\ac{CLI} and logs generated by the simulator are also valid alternatives for observation,
even though the \ac{UX} is worse compared to a \ac{GUI}.

Moreover,
\ac{AR} has been successfully used to provide additional feedback to users,
such as tactile sensations,
making simulations feel more realistic~\cite{Jud2020}.
%
Interaction can be conducted using external hardware such as mouse, keyboard, touch screen, or controllers,
and could require sensors and actuators depending on the context of the simulation.

At the same time,
\ac{LLM}s are emerging as new powerful tools to assist users in reaching their goals in very diverse fields,
spanning from code generation,
to assistance to medical doctors~\cite{Wu2024}.
%
At the time of writing,
\acp{LLM} have also been identified as a useful tool to generate simulation scenarios~\cite{Zhang2023}.
%
However,
the use of \ac{LLM} in the context of simulations is still an open field,
lacking cases where \ac{LLM} are used to assist users in the whole process of interacting with a simulator.

This paper proposes methods to enhance simulator interaction using \acp{LLM},
discussing how such tools can be used to both configure simulations and interact with them.
%
Moreover,
the Alchemist Simulator~\cite{Pianini_2013} will be used as a case study to explore the evolution of \acp{HMI} in simulations and to discuss the potential benefits of employing \acp{LLM} to assist users in interacting with simulations.

\subsection{The Evolution of \acp{HMI} and the impact on Simulators}

\acp{HMI} have significantly evolved over the past decades,
starting from push buttons as ways to interact with machines and light indicators as a way of providing feedbacks.
%
With the advent of \ac{PC},
keyboards quickly became the primary tools for interaction.
%
Display technology evolved accordingly,
introducing \acp{GUI} as an intuitive way to visualize machines' state.
%
Finally,
the rise of the web introduced modern standards that enhanced accessibility and usability of the services aboard machines.
%
Nowadays,
it is common for services to be as intuitive as possible,
so that users can understand how to use them in a matter of seconds.
%
The advent of mobile devices also introduced new interaction methods,
such as touch screens,
enabling users to use their own fingers as input devices.
%
Simulators are no exception to this evolutionary trend.
%

The general-purpose Alchemist Simulator serves as a case study to examine the evolution of its interaction methods.
%
The interaction with the simulator was initially introduced through a \ac{CLI},
which could be used for both configuration and visualization.
%
Gradually,
a desktop-based \ac{GUI} was introduced to provide a more intuitive way for interaction.
%
The introduction of a visual representation of the simulation greatly enhanced the \ac{UX} and the speed at which users could interact with the simulator.
%
The desktop-based \ac{GUI} is currently being migrated to a web-based \ac{GUI} to increase \emph{usability},
allowing users to interact with simulations using a web browser on any device,
including smartphones and tablets.
%
Additionally,
implementing new \emph{functionalities} becomes easier,
since there are many tools oriented to data visualization available on the ecosystem.
%

\subsection{Introducing \ac{LLM} in Simulations}
Software typically serves as a mean to achieve a goal.
%
For example,
an \ac{IDE} serves as a tool to write source code from scratch.
%
Nowadays,
it is common for \acp{IDE} to include tools like GitHub Copilot\footnote{
    GitHub Copilot · Your AI pair programmer, \url{http://archive.today/jsEIH}
} to further assist users in writing code.
%
Such software can successfully be used as effective pair programming tools by experienced developers~\cite{DBLP:journals/jss/DakhelMNKDJ23}.
%
On the other hand,
simulations are also software which enable the visualization of a designed and defined representation of processes.
%
\begin{figure}
    \includegraphics[width=0.954\columnwidth]{use-case}
    \caption{
        Comparison between tools used in code generation and simulation.
        %
        The lower block symbolizes the tool that a user utilizes to achieve the goal,
        which is represented by the upper block.
        %
        The blocks in the middle represents the additional tool based on \ac{LLM} that can be used to assist the user.
    %
    }
    \label{fig:usecase}
\end{figure}
%
\Cref{fig:usecase} shows a comparison between the usage of an \ac{LLM} in code generation and a possible use case in simulations.
%
Copilot is used to assist the user in writing code,
and it is able to generate pieces of code that fulfill the user's needs.
%
The same idea could be applied to simulations,
where an \ac{LLM} can be used to assist the user in configuring the simulator and even interact with it.
%
In fact,
configuring a simulator by hand can be time-consuming,
especially for new users that are not familiar with the domain model and the semantics.
%
Ideally,
a \ac{LLM} could become a real assistant for the user,
not limited to the configuration of the simulator,
providing multiple functionalities related to the interaction with the simulator.
%
\acp{GUI} would still remain the primary way to visualize simulations,
while \acp{LLM} could still be used as a tool to quickly customize the appearance based on the user's needs.

As an example,
imagine a complex scenario of a distributed simulation with a pool of network nodes available.
%
An inexperienced user wants to create a number of differently parameterized simulations,
and run one of each per node.
%
The user would firstly need to understand the domain,
then configure the simulator,
then understand how to run a the simulations and finally link a \ac{GUI} to visualize the simulation progress.
%
The inclusion of an \ac{LLM} would not only help the user to solve problems related to the domain,
but would also enable the user to correctly configure, run and visualize the simulation in the way described,
if possible.
%

There are plenty of models and techniques available to apply \ac{LLM},
based on specific use cases.
%
One example that could suit the simulation use case is the ReAct prompting technique~\cite{DBLP:conf/iclr/YaoZYDSN023},
which focuses on generating both reasoning traces and task-specific actions.
%
Moreover,
the framework is also able to retrieve information from external environments,
enabling for avoidance of fact hallucination.
%
However,
the best way to design of an \ac{LLM} depends on a variety of factors,
and research still needs to be properly conducted.

\subsection{Conclusion and Future Works}
%
The future of \acp{HMI} in simulations is still an open field.
%
\ac{LLM} and similar technologies are promising and pervasive tools that can be used to assist users in solving everyday problems,
and simulations are no exception.
%
The inclusion of such tools in simulations could result to be very beneficial in terms of \ac{UX},
but research still needs to be conducted in order to understand how to effectively design a \ac{LLM} that can assist users in interacting with simulations.
%
There are several possible way to effectively design a \ac{LLM} in order to improve the prompts and get better results,
and work will be conducted to find the best way to interact with simulators.

\bibliographystyle{IEEEtran}
\bibliography{doctoral-colloquium-2024-dsrt-simulation-interaction}
\vspace{12pt}
\end{document}


